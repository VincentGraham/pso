\documentclass[journal, onecolumn, 12p]{IEEEtran}
\usepackage{fontspec}
\usepackage{microtype}
\usepackage{graphicx}
\usepackage{amssymb, amsmath, amsfonts}
\usepackage[left=1in,top=2cm,right=1in,nohead]{geometry}
\usepackage{subfiles}
\usepackage{color, soul}
\usepackage{esdiff}
\usepackage{physics}

\usepackage{latexsym}
\usepackage{epsfig,float,subcaption}

\usepackage[russian, main=english]{babel}
\babelfont{rm}{FreeSerif}


\usepackage{mathbbol}
\DeclareSymbolFontAlphabet{\mathbb}{AMSb}
\DeclareSymbolFontAlphabet{\mathbbl}{bbold}


\renewcommand{\vec}{\mathbf}
\usepackage{listings}
\usepackage[final]{matlab-prettifier}
\lstset{
    style=Matlab-editor,
    basicstyle=\ttfamily
}

\usepackage{fancyhdr}
\fancyhf{}

\cfoot{\thepage}
\pagestyle{plain} 


% some from https://tex.stackexchange.com/questions/111116/what-is-the-best-looking-pseudo-code-package
\newcounter{alg}[section] % defines algorithm counter for chapter-level
\renewcommand{\thealg}{\thesection .\arabic{alg}}
\lstnewenvironment{algorithm}[1][] %defines the algorithm listing environment
{   
    \refstepcounter{alg} %increments algorithm number
    \lstset{  
        firstnumber=last,
        escapechar=!,
        mathescape=true,
        frame=tB,
        numbers=left, 
        numberstyle=\tiny,
        basicstyle=\scriptsize\linespread{1.3}\selectfont, 
        keywordstyle=\color{black}\bfseries\em,
        keywords={,input, output, return, datatype, function, in, if, else, foreach, while, begin, end, for, sort, is, initialize, } %add the keywords you want, or load a language as Rubens explains in his comment above.
        numbers=left,
        xleftmargin=.04\textwidth,
        #1 % this is to add specific settings to an usage of this environment (for instnce, the caption and referable label)
    }
}
{}

\usepackage{emoji}
\newcommand{\R}{{\mathbb R}}
\date{}


\fontdimen16\textfont2 = 9\fontdimen16\textfont2
\fontdimen16\scriptfont2 = 9\fontdimen16\scriptfont2 
\fontdimen17\textfont2 = 9\fontdimen17\textfont2
\fontdimen17\scriptfont2 = 9\fontdimen17\scriptfont2 

\let\oldemoji\emoji
\renewcommand{\emoji}[1]{\ensuremath{\smash{\vcenter{\hbox{\scalebox{1.1}[1.2]{\oldemoji{#1}}}}}}}
\newcommand{\bigemoji}[1]{\scalebox{1.3}[1.35]{\oldemoji{#1}}}
\newcommand{\smallemoji}[1]{\scalebox{.8}[.87]{\oldemoji{#1}}}
\newcommand{\elephant}{\ensuremath{{\emoji{elephant}\kern-.09em}}}
% \newcommand{\matriarch}{\emoji{mammoth}}
\newcommand{\herd}{\smash{\vcenter{\hbox{\kern-.05em\bigemoji{elephant}\kern-.6em\smallemoji{elephant}\kern-.55em\bigemoji{elephant}\kern-1em\bigemoji{elephant}\kern-.7em\bigemoji{elephant}\kern-.09em}}}}
% \newcommand{\population}{\herd\kern-2em\herd}
\newcommand{\population}{\ensuremath{\{\herd_1, \herd_2,\ldots, \herd_m\}}}

% \newcommand{\forest}{\smash{\vcenter{\hbox{\kern-.05em\bigemoji{deciduous-tree}\kern-.6em\smallemoji{deciduous-tree}\kern-.55em\bigemoji{deciduous-tree}\kern-1em\bigemoji{deciduous-tree}\kern-.09em}}}}

\DeclareMathOperator*{\argmin}{\arg\!\min}
\DeclareMathOperator*{\dom}{dom}

\newcommand{\mathcms}[1]{\text{\fontspec{Comic Neue}#1}}
\newcommand{\mathcmsbf}[1]{\textbf{\fontspec{Comic Neue}#1}}

\setemojifont{AppleColorEmoji.ttf}

\begin{document}
\author{Vincent Graham}
\title{Data Clustering with Elephant Herding Optimization}
\maketitle
\thispagestyle{plain}

\begin{abstract}
Clustering is a data analysis task that groups objects using a similarity measure. In this paper, I will present an overview of a clustering algorithm based on the herd structure and behavior of the African Savanna Elephant. This will be done using biological analogy and several derivations from other iterative optimization strategies. A gradient-based version of elephant herding optimization will also be discussed and compared against the original on several data sets. A more detailed example is shown for the Fisher Iris data set. This paper will provide a MATLAB implementation of both algorithms.
\end{abstract}

% \subfile{algorithm.tex}

\subfile{introduction.tex}
\subfile{eho.tex}
\subfile{derivations.tex}
\subfile{results.tex}
\nocite{*}
\clearpage

\bibliographystyle{plain}
\bibliography{elephants.bib, introduction.bib}
\newpage
\clearpage

\appendix
\subsection{Particle swarm optimization}
\lstinputlisting{code/pso.m}
\clearpage
\lstinputlisting{code/pso_1d.m}
\clearpage
\subsection{Elephant herding optmization}
\lstinputlisting{code/eho.m}
\clearpage
\subsection{Gradient based elephant herding optimization}
\lstinputlisting{code/gbeho.m}
\clearpage
\lstinputlisting{code/LEO.m}
\clearpage
\subsection{Other}
\lstinputlisting{code/fitness.m}
\clearpage

\end{document}
